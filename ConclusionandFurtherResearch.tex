\begin{savequote}[10cm] % this sets the width of the quote
\sffamily
Lorem ipsum dolor sit amet, consectetur adipiscing elit, sed do eiusmod tempor incididunt ut labore et dolore magna aliqua. Ut enim ad minim veniam, quis nostrud exercitation ullamco laboris nisi ut aliquip ex ea commodo consequat. Duis aute irure dolor in reprehenderit in voluptate velit esse cillum dolore eu fugiat nulla pariatur. Excepteur sint occaecat cupidatat non proident, sunt in culpa qui officia deserunt mollit anim id est laborum.
\qauthor{An Author}
\end{savequote}


\chapter{Conclusion and Further Research}

Social and communication technology has changed and will continue to change Australia and Australians and a comparative study at a future time would be instructive as to exactly how much.

Much of technology policy is based around a model where government and businesses simply download policy onto its constituents without debate, dialogue or much less imagining a new system of networked information structures in concert with the users of that system. This may be due to the fact that wide consultation with a constituent population has been expensive, difficult and time consuming in the past but now the internet itself and social networking has made it much easier to present, evaluate and collect ideas from people even in remote and very remote areas.

Technological change is accelerating and the number of elements connected to the social/communication network is increasing, possibly exponentially, even in regional and remote Australia. As technologies such as ubiquitous internet connectivity and the Internet of Things become commonplace, more elements of the Australian environment and society that are currently outside the internet will be included in it. Using social technologies, dependencies and connections that were previously invisible will be revealed and become a platform for faster and more comprehensive change than has been possible before. 



Changes to cultural and social systems are inevitable with the introduction of technology to communicate and organise our society and culture. The question would be would it be better to integrate the technology quickly into a society or would gradual introduction or even segregation traditional society and culture from technology be more advisable. Will cultural archives be better housed electronically than within existing cultural systems? What are the privacy implications of more private data such as the MyHealthRecord being housed electronically and have privacy data breaches been materially worse for Australians in the digital realm compared to the current system of paper and electronic records? How will technology change the fabric of Indigenous and non-Indigenous culture and society and should further research be conducted in this regard?


\section{Projections of NBN Growth}
The NBN already covers Australia and it's territories through combination of wired and wireless technologies. The future of the NBN rests in its continued support by the Australian Government and at this stage there is no indication of that being withdrawn ahead of a a possible sell of of the network at some time after it is completed between 2020 and 2025.

\section{Further technology advances that may increase internet reach}
At this stage there seems to be no end to increase in data consumed by each individual on the internet. This annual data doubling, described epynomyously as `Zuckerberg's law' by the Facebook founder, says that each year the network will need to carry twice as much traffic as before as people and their devices share more and more information over this ubiquitous network.

New technology such as 8K television and video games, which is on many technology road maps post 2020\cite{RefWorks:445}, will contribute to this data demand, as will the increase in the number of personal network devices being carried by Australians as they travel around the network on their daily business. Autonomous vehicles and increasing density of mobile networks in Australia's urban centres will require a move to 5G technology to increase density and improve communications fidelity while reducing latency that currently makes autonomous vehicles more safe and allows them to use the road network more efficiently by packing more tightly together\cite[p19]{RefWorks:445}.

The current NBN is undergoing a speed upgrade and 1Gbps connections will soon be available to some consumers. This is well behind other markets where multi-gigabit connections to the home have been available for some time and there is a possibility that telecommunications carriers may choose to sidestep wired technologies and implement gigabit connections to some price insensitive and high value customers over 5G instead of using fibre. The NBN itself is experimenting with using 5G fixed wireless\cite[p12]{RefWorks:444} as they currently use 4G LTE so this innovation may end up being widely distributed relatively quickly in remote and rural locations for a fixed cost even as it is taken up by technology elites in urban areas where the service will be subject to user pays charging.

\section{Aboriginal people as part of the Remote, Rural and Regional populations}
While Indigenous people represent around 3\% and climbing of the Australian population, they represent around 20\% of the residents of remote and very remote areas.


\begin{quotation}
Indigenous Australians, as they are more likely to live outside metropolitan areas than non-Indigenous Australians. In 2011, just over one third of Indigenous Australians lived in Major cities (34.8\%), compared with over 70\% of non-Indigenous
Australians. Only 1.7\% of non-Indigenous Australians lived in Remote or Very remote areas, compared with
about one-fifth of Indigenous Australians (7.7\% in Remote and 13.7\% in Very remote areas). Indigenous
Australians represent 16\% and 45\% of all people living in Remote and Very remote areas respectivel.\cite{RefWorks:446}.
\end{quotation}

When I started work on this thesis there was evidence that Indigenous people were migrating from urban centres to regional, remote and very remote areas. It would be interesting to study this in the context of whether the assistance of social media and improved communications systems from the NBN facilitated this movement for Indigenous people specifically due to enhanced cultural and social support it gives.

